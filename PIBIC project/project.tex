\documentclass[11pt,a4paper]{article}
% This text is inserted in the beginning of all
% LaTex and Tex files I create.
%
% File created: Tue Sep 26 2017
% File name:    report_template.tex
% Path:         /home/name/Classes/AMPIII/Template/
%
% Name
% Sept, 2017
%

\usepackage[brazilian]{babel}
\usepackage[utf8]{inputenc}

%recommended by fancyhdr package
\setlength{\headheight}{14.49998pt}
\addtolength{\topmargin}{-2.49998pt}

% include a minimal set of useful packages
\usepackage{graphicx}
\usepackage{amsfonts} 
\usepackage{amssymb}
\usepackage{amsmath}
\usepackage[a4paper,left=3.0cm,top=3.0cm,right=2.0cm,bottom=2.0cm]{geometry}
\usepackage{lastpage}
\usepackage{fancyhdr}
\usepackage{natbib,stfloats}
\usepackage[parfill]{parskip}
\usepackage{multirow}
\usepackage{helvet}

% PUT YOUR TITLE AND NAME HERE
\newcommand{\titlestr}{Identificão de padrões de demanda para uma empresa do setor de plásticos \\ Projeto PIBIC}
\newcommand{\shorttitlestr}{Identificação de padrões de demanda...}
\newcommand{\authorstr}{Profa. Tatiana Balbi Fraga} % INSERT YOUR NAME(S)

\begin{document}
%%%%%%%%%%%%%%%%%%%%%%%%%%%%%%555
% title page
\begin{titlepage}
  \centering
  
  {\LARGE \titlestr \par}

  \vspace{1cm}
  {\Large \authorstr \par}

  {\bf Aluna: Beatriz Marinho Cavalcanti}

  \vspace{1cm}
  \today     % PUT YOUR DATE HERE

  \vspace{1.5cm}
  Projeto submetido ao
  {\bf Edital PROPESQI no 02/2022}
  da Pró-Reitoria de Pesquisa e Inovação
  da Universidade Federal de Pernambuco

  \includegraphics[width=0.25\textwidth]{logos/ufpe_logo.png}

  \vspace{2cm}
  \flushleft
  {\bf Departamento de Engenharia de Produção} \\
  {\bf Núcleo de Tecnologia} \\
  {\bf Centro Acadêmico do Agreste} \\

  \vspace{5mm} {\footnotesize Ao submeter este projeto eu declaro que todo o material aqui compreendido é fruto de meu próprio trabalho e todas as referências consultadas estão claramente citadas.
    Este projeto foi elaborado após extenso trabalho de pesquisa realizado por mim (incluindo levantamento bibliogŕafico e compreensão - apresentação da metodologia a ser aplicada), como parte projeto de pesquisa registrado na Propesqi \citep{Fraga2019}.\par}
    
  \vspace{2mm} {\footnotesize O texto do projeto (atualizado), está disponível em forma de artigo no site MadScientistsDiary: https://github.com/tbfraga/gamos-madScientistsDiary-, portanto qualquer utilização, distribuição, replicação deste projeto deve estar em conformidade com a Licença Pública Internacional Creative Commons Atribuição-NãoComercial-SemDerivativos 4.0 e qualquer outro projeto desenvolvido com base neste projeto deve citá-lo como referência respeitando os direitos de autoria.\par}
    
  \vfill
\end{titlepage}

% put headings on each page
\pagestyle{fancy}
\fancyhf{}
\rhead{\shorttitlestr}
\lhead{\authorstr}
\rfoot{Page \thepage\ of \pageref{LastPage}}
\renewcommand{\headrulewidth}{1pt}


%%%%%%%%%%%%%%%%%%%%%%%%%%%%%%555
% main report
\clearpage
\section{Introdução}

A previsão de demanda é essencial para o bom planejamento em qualquer empresa. Através de uma boa previsão é possível, entre outras coisas, controlar melhor os níveis de estoque, reduzindo custos e oferecendo um melhor nível de serviço aos clientes. 

Conforme mostra \cite{MakridakisHibon2000}, a literatura apresenta uma grande variedade de metodologias que podem ser utilizadas para previões de demanda, sendo que a performance dos distintos modelos de previsão varia de acordo a natureza dos dados e um modelo que gera bons resultados para determinada classe de itens de uma empresa pode gerar previsões ruins para outros itens dessa mesma empresa.

Uma estratégia natural utilizada para identificar o modelo de previsão adequado pra cada item consiste em comparar a performance dos distintos modelos candidatos utilizando dados históricos de vendas do item \citep{UlrichEtAl2022}. Contudo, como geralmente as empresas produzem e/ou comercializam uma grande variedade de itens, essa estratégia acaba se tornando um esforço considerável. 

De acordo com \cite{UlrichEtAl2022}, uma opção viável consiste em agrupar os itens de acordo com seus padrões de demanda, para posteriormente identificar o modelo de previsão adequado pra cada grupo e não mais para cada item individual. O projeto elaborado por \cite{Fraga2019} propõem uma abordagem similar, contudo busca a identificação de padrões de demanda para os principais itens produzidos e/ou comercializados por um grupo de micro e pequenas empresas do Agreste Pernambucano, visando o desenvolvimento de uma metodologia combinada que seja adequada a um cojunto de padrões de demanda distintos e recorrentes nestas empresas. Como parte do projeto proposto por \cite{Fraga2019}, este projeto PIBIC busca identitificar os padrões de demanda dos principais produtos de uma empresa do setor de plásticos. Para tanto serão considerados os seguintes objetivos:

\begin{itemize}
  \item identificação de sistema de armazenamento de dados utilizado pela empresa;
  \item identificação de metodologia atualmente aplicada para previsão de demanda na empresa;
  \item identificação dos principais produtos; 
  \item levantamento de dados (históricos de vendas e outros dados relevantes);
  \item compreensão de metodologias aplicadas para identificação de padrão de demanda; e
  \item análise dos padrões de demanda.
\end{itemize}
  
Os dados coletados neste projeto, assim como os estudos realizados serão de grande importância científica, tendo em vista que serão utilizados para o desenvolvimento de uma nova metodologia de previsão de demanda combinada, e também poderão ser utilizados para outros trabalhos futuros relacionados ao setor estudado, tornando se referência para diversos estudos que venham a ser desenvolvidos.

\section{Fundamentação Teórica}

'Os padrões de demanda são resultados da variação da demanda com o tempo, ou seja, do crescimento ou declínio de taxas de demanda, sazonalidades e flutuações gerais causadas por diversos fatores' (\cite{Ballou2001} apud \cite{WernerEtAl2006}). 

De acordo com \cite{Ballou2006}, quando a demanda apresenta comportamento regular, os padrões de demanda podem ser divididos em compo­nentes de tendência, sazonais ou aleatórios. Já nos casos em que a demanda de determinados itens é inter­mitente, em função do baixo volume geral e da incerte­za quanto a quando e em que nível essa demanda ocor­rerá, a série de tempo é chamada de incerta, ou irregu­lar.

\cite{BoylanEtAl2008} distribuem os padrões de demanda entre normais, onde a demanda pode ser representada por uma distribuição normal, e não normais, no caso em que isso não é possível. De acordo com os autores, os padrões de demanda não normais podem ser classificados da seguinte forma:

\begin{itemize}
  \item um item de \emph{demanda intermitente (intermittent)} é um item com ocorrências de demanda pouco frequêntes;
  \item um item de \emph{demanda de movimento lento (slow moving)} é um item cuja demanda média por período é baixa. Isso pode ser devido a ocorrências de demanda pouco frequentes, tamanhos médios de demanda baixos ou ambos;
  \item um item de \emph{demanda errática (erratic)} é um item cujo tamanho de demanda é altamente variável;
  \item um item de \emph{demanda esporádica (lumpy)} é um item intermitente para o qual a demanda, quando ocorre, é altamente variável.; e
  \item um item de \emph{demanda agregada (clumped)} é um item intermitente para o qual a demanda, quando ocorre, é constante (ou quase constante).
\end{itemize}

Apesar da importância da identificação dos padrões de demanda para identificação dos métodos adequados de previsão, poucos autores tratam deste assunto e poucas técnicas são apresentadas na literatura para essa finalidade.

\cite{BusingerRead1999} aplicam um sistema de agrupamento de itens utilizando um diagrama de plotagem em estrela considerando oito características dos dados de séries temporais: coeficiente de variação, número de zeros, tendência, picos (outliers), sazonalidade, corridas, assimetria e autocorrelação.

O coeficiente de variação $(CV)$ informa a variabilidade em relação à média. Essa medida adimensional informa o nível de dispersão: quanto mais alto for o $(CV)$, mais alta é a dispersão:

\begin{equation}
CV = \frac{s}{\bar{i}}
\end{equation}

onde: 

$\bar{i}$ é a média dos valores considerados, e

$s$ é o desvio padrão 

\begin{equation}
s = \sqrt{ \frac{1}{N} \sum_{i=1}^{N}{(i-\bar{i})^2}}
\end{equation}

O número de zeros $(NZ)$ é uma medida que indica o número de períodos com demanda zero (nula) dentro de determinado intervalo de tempo. 

\begin{equation}
NZ = \sum_{i=1}^{N}{I(y_i=0)}
\end{equation}

Onde $I(A)=1$, se $A$ é verdadeiro, e $I(A)=0$, se $A$ é falso.

Essa medida é, em alguns casos, associada à intermitência. Sendo $BP$ um valor de corte, se $(NZ\geq BP)$ então a demanda é considerada intermitente \citep{BoylanEtAl2008}.

A tendência $(T)$ apresenta um padrão de variação suave e temporário na demanda. Para cálculo da tendência, \cite{BusingerRead1999} dividem os dados avaliados em terços e calculam a tendência usando a seguinte expressão:

\begin{equation}
T = \frac{(Y_U - Y_L)}{(Y_{(\frac{5}{6})} - Y_{(\frac{1}{6})})}
\end{equation}

Onde $Y_U$ e $Y_L$ representam as medianas dos terços extremos, sendo $L$ o terço inferior, e $U$ o terço superior. Observe que $-1 \geq T \leq 1$.

Picos $(P)$ é uma característica que informa sobre padrões nos dados temporais que podem representar anomalias ou dados súbitos.

\begin{equation}
P = \sum_{i=1}^{N}{I(d_i > 2)}
\end{equation}

onde:

\begin{equation}
d_i = \frac{y_i - y_{T}}{s_{T}}
\end{equation}

sendo $y_{T}$ e $s_{T}$, respectivamente, a média e o desvio padrão aparados (i.e., após retirar 20\% dos dados da amostra, sendo 10\% referente aos menores valores e 10\% referente aos maiores valores).

A sazonalidade informa comportamentos que se repetem a cada ciclo (normalmente de 3 meses). \cite{BusingerRead1999} represetam a sazonalidade através da seguinte medida adimensional:

\begin{equation}
SS = 1 - \frac{ss_{w}}{ss_{T}}
\end{equation}

onde:

\begin{equation}
ss_{w} = \sum_{i=1}^{4}\sum_{j=i}^{n_i}{x_{ij}}
\end{equation}

e

\begin{equation}
ss_{T} = \sum_{i}\sum_{j}{(x_{ij}-\bar{x})^2}
\end{equation}

sendo:

\begin{equation}
\bar{x_i} = \frac{1}{n_i} \sum_{j=1}^{n_i}{x_{ij}}
\end{equation}

e

\begin{equation}
\bar{x} = \frac{1}{n} \sum_{i=1}^{4}{n_i \bar{x_i}}
\end{equation}

tal que: 

$y_{i}^w$ representa os dados winsorizados (i.e. após retirar 20\% dos dados da amostra, sendo 10\% referente aos menores valores e 10\% referente aos maiores valores, substituindo esses dados, respectivamente, pelo menor e pelo maior valor dentro do intervalo de dados restantes (\%80 dos dados)); 

$n=4k+r$ representa o número total de períodos para $r = 0, 1, 2, 3$, sendo o período $(n_i)$, é definido por:

\begin{equation}
n_i = 
\begin{cases}
4k, \ \quad \mathrm{se} \quad r =0 \\
4k + i, \quad \mathrm{se} \quad r > 0
\end{cases}
\end{equation}

\begin{equation}
x_{ij} = y_{i+4(j-1)}^w \quad i = 1,...,4 \quad j=1,...,k
\end{equation}

e

\begin{equation}
x_{i(k+1)} = y_{n_i}^w \quad i = 1,...,r \quad r>0
\end{equation}

De acordo com \cite{BusingerRead1999}, a característica corrida representa um intervalo no qual observações sucessivas ocorrem todas no mesmo lado da mediana do processo. \cite{BusingerRead1999} utilizam o número de corridas acima e abaixo da mediana. Os autores afirmam que um valor baixo para o número total de corridas pode confirmar uma tendência; enquanto que um valor alto sugere agrupamento ou outro comportamento oscilatório.

\begin{equation}
Run = \frac{|R - E[R]|}{\sqrt{Var(R)}}
\end{equation}

onde: 

R representa a soma dos números de corridas abaixo e acima da mediana (após exclusão dos valores iguais à mediana),

\begin{equation}
E[R] = 1 + \frac{2n_1n_2}{n_1 + n_2}
\end{equation}

\begin{equation}
Var(R) = \frac{2n_1n_2(2n_1n_2 - n_1 - n_2)}{(n_1 + n_2)^2(n_1+n_2-1)}
\end{equation}

sendo:

$n_1$ o número total de observação acima da mediana; e

$n_2$ o número total de observação abaixo da mediana.

A assimetria $(A)$ é a razão entre a média aritmética $(\bar{y})$ e a mediana $(m)$.

\begin{equation}
A = \frac{\bar{y}}{m}
\end{equation}

Valores de $A$ maiores que 1, indicam uma assimetria positiva, e valores de $A$ menores do que 1, indicam uma assimentria negativa.

A correlação $CR$ indica a relação entre dois semestres subsequentes:

\begin{equation}
CR = \frac{|RVN-2|}{\sigma}
\end{equation}

onde:

\begin{equation}
\sigma^2 = \frac{4(n-2)(5n^2-2n-9}{5n(n+1)(n-1)^2}
\end{equation}

\begin{equation}
RVN = \frac{12NM}{n(n^2-1)}
\end{equation}

\begin{equation}
NM = \sum_{i=1}^{n-1}{[r_i-r_{i+1}]^2}
\end{equation}

$r_i$ representa o rank de $y_i$.

\cite{Williams1984} apresenta um esquema para classificação da demanda em suave (smooth), de movimento lento, ou esporádica, particionando a variabilidade da demanda durante um lead time $(C_{LTD}^{2})$ em suas partes causais constituintes: variabilidade dos números de pedidos $(\frac{C_{n}^{2}}{\bar{L}})$, variabilidade dos tamanhos dos pedidos $(\frac{C_{x}^{2}}{\bar{n}\bar{L}})$ e variabilidade dos prazos de entrega $(C_{L}^{2})$. 

\begin{equation}
C_{LTD}^{2} = \frac{C_{n}^{2}}{\bar{L}} + \frac{C_{x}^{2}}{\bar{n}\bar{L}} + C_{L}^{2}
\end{equation}

onde:

$n$ representa os números de pedidos que chegam em unidades de tempo sucessivas (variáveis randômicas independetes e identicamente distribuídas (IIDRVs), com média $\bar{n}$ e variância $var(n)$), \\

$x$ representa os tamanhos dos pedidos (IIDRVs, com média $\bar{x}$ e variância $var(x)$),  \\

$L$ representa os prazos de entrega (IIDRVs, com média $\bar{L}$ e variância $var(L)$)), e \\

$C_{i}$ representa o coeficiente de variação de $i$. \\

Tal esquema foi posteriomente revisado por \cite{EavesKingsman2004}, considerando também o padrão de demanda irregular. A classificação adaptada por pelos autores é apresentada na Tab. 1.

\begin{table}[h]
\begin{center}
\begin{tabular}[c]{c c c c}
\cline {1-4}
$\frac{C_{n}^{2}}{\bar{L}}$ & $\frac{C_{x}^{2}}{\bar{n}\bar{L}}$ & $C_{L}^{2}$ & padrão de demanda \\ \cline {1-4}
baixo & baixo &  & suave   \\ 
baixo & alto  &  & irregular   \\ 
alto  & baixo &  & de movimento lento   \\
alto  & alto  & baixo & intermitente   \\
alto  & alto  & alto  & atamente intermitente\\ \cline {1-4}
\end{tabular}
\label{tab:DemandPattern}
\caption{Classificação dos padrões de demanda de acordo com \cite{EavesKingsman2004}.}
\end{center}
\end{table}

Observe que os valores dos critérios de corte definidos para diferenciar alto e baixo são arbitrários.

\cite{SyntetosEtAl2005} sugerem um esquema de categorização contruído a partir da comparação do erro médio quadrado de três diferentes metodologias (método de Croston, método de Croton modificado e amortecimento exponecial simples). De acordo com esse esquema, os parâmetros quadrado do coeficiente de variação do tamanho da demanda $(CV^2)$ e intervalo médio entre demandas $(p)$ são usados para classificar a demanda entre errática, esporádica, suave e intermitente. A tabela a seguir apresenta a classificação proposta pelos autores.

\begin{table}[h]
\begin{center}
\begin{tabular}[c]{c c c}
\cline {1-3}
$CV^2$ & $p$ & \multirow{2}{*}{padrão de demanda} \\ 
$0.49$ & $1.32$ & \\ \cline {1-3}
baixo & baixo & suave   \\ 
baixo & alto  & intermitente   \\ 
alto  & baixo & errática   \\
alto  & alto  & esporádica  \\ \cline {1-3}
\end{tabular}
\label{tab:DemandPatternSybtetos}
\caption{Classificação dos padrões de demanda de acordo com \cite{SyntetosEtAl2005}.}
\end{center}
\end{table}

Observe que nesse esquema de classificação são definidos os pontos de corte $CV^2=0.49$ e $p=1.32$.

\section{Metodologia}

Este projeto será desenvolvido através de 5 atividades, conforme descrito a seguir:\\

Atividade 1: identificação do sistema de dados e da metodologia de previsão de demanda utilizados na empresa - através de conversas com funcionários das empresas.\\

Atividade 2: identificação dos principais produtos - através da metodologia de classificação ABC (utilizando dados históricos dos últimos meses para todos os produtos).\\

Atividade 3: levantamento de dados históricos de vendas e dos últimos quatro anos (quando disponíveis) e de outras informações necessárias para os princiapis produtos (de acordo com classificação ABC).\\

Atividade 4: compreensão das metodologias aplicadas para identificação de padrões de demanda, conforme descritas na segunda seção.\\

Atividade 5: aplicação dos metodologias de identificação de padrões de demanda para os princiapis produtos e análise dos resultados.\\

Atividade 6: preparação de relatórios e artigos para o CONIC.

Tais atividades serão desenvolvidas pela aluna, sob a orientação e supervisão da Profa. orientadora (autora deste projeto).

\section{Resulados esperados}

Após a aplicação da metodologia acima descrita, esperamos esperamos obter os seguintes resultados: 

\begin{itemize}
    \item dados de históricos de vendas e outros dados relevantes referentes aos principais produtos da empresa estudada;
    \item mapeamento dos padrões de demanda destes produtos;
    \item aprofundamento do conhecimento dos participantes dos projetos sobre os comportamentos de vendas e sobre a identificação de padrões de demanda;
    \item artigo apresentado no CONIC.
\end{itemize}

Os resultados destes trabalhos também serão incluídos em pelo menos um artigo que será submetido para revista.

\section{Viabilidade de execução}

Os projetos serão realizados preferencialmente no CAA-UFPE. Assim, todos os participantes deste projeto, terão acesso garantido a toda a infraestrutura necessária para o correto desenvolvimento de seu trabalho incluindo recursos físicos (sala e mobiliário), bibliográficos e computacionais da UFPE e mais especificamente do CAA. O departamento de engenharia de produção do CAA conta atualmente com dois laboratórios de informática que disponibilizam, pelo menos, 30 computadores. O GAMOS, em especial, conta com laboratório próprio, e que atualmente dispõe de 3 computadores. Em termos de recursos bibliográficos, os pesquisadores da área de engenharia da produção contam com a biblioteca central da UFPE e as bibliotecas setoriais do CTG (Centro de Tecnologia e Geociências) e do CCEN (Centro de Ciências Exatas e da Natureza), localizadas no campus da UFPE de Recife, e com a biblioteca do próprio CAA, que possuem assinatura de alguns dos principais periódicos na área além do acesso remoto à base de dados disponíveis hoje via rede entre elas o banco de dados disponibilizados pela CAPES e pelo sciencedirect. 

Parte do projeto será também realizada através de visitas à empresa escolhida para realização desse trabalho. A parceria com a empresa será firmada durante a realização do projeto e o termo de parceria será anexado no relatório final.

\section{Cronograma de atividades do estudante}

Este projeto foi planejado para ser realizado durante o período de 1 ano, com início previsto para 2022/2023. As atividades descritas na metodologia estão projetadas para serem realizadas conforme cronograma apresentado na Tab. 3 :

\begin{table}[h]
\begin{center}
\begin{tabular}[c]{||c||c|c|c|c|c|c||}
\cline {1-7}
\multirow{2}{*}{Atividade} & \multicolumn{6}{c ||}{Cronograma (bimestre)} \\ \cline {2-7}
 & $1^o$ & $2^o$ & $3^o$ & $4^o$ & $5^o$ & $6^o$ \\ \cline {1-7}
$1^a$ & xx &  &  &  &  &  \\
$2^a$ & xx &  &  &  &  &  \\
$3^a$ &  & xx & xx &  &  &  \\
$4^a$ &  &  & xx & xx & xx &   \\
$5^a$ &  &  & xx & xx & xx & xx  \\
$6^a$ &  &  &  &  &  & xx  \\ \cline {1-7}
\end{tabular}
\label{tab:Cronograma}
\caption{Cronograma planejado para o projeto.}
\end{center}
\end{table}

%\bibliographystyle{plain}
%\bibliography{bib_file}

\clearpage
\begin{thebibliography}{12}

\bibitem[\protect\citeauthoryear{Ballou}{2001}]{Ballou2001}
Ballou, R.H. (2001).{\it Gerenciamento da Cadeia de Suprimentos: Planejamento, Organização e Logística Empresarial}, 4. ed., Porto Alegre: Bookman.

\bibitem[\protect\citeauthoryear{Ballou}{2006}]{Ballou2006}
Ballou, R.H. (2006).{\it Gerenciamento da Cadeia de Suprimentos / Logística Empresarial}, 5. ed., Porto Alegre: Bookman.

\bibitem[\protect\citeauthoryear{Boylan et al.}{2008}]{BoylanEtAl2008}
Boylan, J.E., Syntetos, A.A., e Karakostas, G.C. (2008). 'Classification for forecasting and stock control:a case study'. {\it Journal of the Operational Research Society}, Vol. 59, pp. 473--481.

\bibitem[\protect\citeauthoryear{Businger e Read}{1999}]{BusingerRead1999}
Businger, M.P., e Read, R.R. (1999). 'Identification of demand patterns for selective processing: acase study'. {\it Omega, Int. J. Mgmt Sci.}, Vol. 27, pp. 189--200.

\bibitem[\protect\citeauthoryear{Eaves e Kingsman}{2004}]{EavesKingsman2004}
Eaves A.H.C., e Kingsman B.G. (2004). 'Forecasting for the ordering and stock-holding of spare parts'. {\it J. O. Opl. Res. Soc.}, Vol. 55, pp. 431--437.

\bibitem[\protect\citeauthoryear{Fraga}{2019}]{Fraga2019}
Fraga, T.B. (2019). 'Estudo de Métodos de Previsão de Demanda e Proposição de Metodologia Combinada no Contexto das Micro e Pequenas
Empresas do Agreste Pernambucano'. Projeto de Pesquisa registrado em 09/11/2019, e aprovado pela Pró-reitoria de Pesquisa da UFPE em 11/02/2021 (Processo SIPAC 23076.057489/2019-21).

\bibitem[\protect\citeauthoryear{Makridakis et al.}{1998}]{MakridakisEtAl1998}
Makridakis, S.G.,Wheelwright, S.C., Hyndman, R.J. (1998).{\it Forecasting: Methods and Applications}, 3. ed., Wiley.

\bibitem[\protect\citeauthoryear{Makridakis e Hibon}{2000}]{MakridakisHibon2000}
Makridakis, S. e Hibon, M. (2000) 'The M3-Competition: results, conclusions and implications'. {\it International Journal of Forecasting}, Vol. 16, pp. 451--476.

\bibitem[\protect\citeauthoryear{Syntetos et al.}{2005}]{SyntetosEtAl2005}
Syntetos, A.A., Boylan, J.E., e Croston, J.D. (2005) 'On the categorization of demand patterns'. {\it Journal of the Operational Research Society}, Vol. 56 (5), pp. 495--503.

\bibitem[\protect\citeauthoryear{Ulrich et al.}{2022}]{UlrichEtAl2022}
Ulrich, M., Jahnke, H., Langrock, R., Pesch, R., e Senge, R. (2022) 'Classification-based model selection in retail demand forecasting'. {\it International Journal of Forecasting}, Vol. 38 (1), pp. 209--223.

\bibitem[\protect\citeauthoryear{Werner et al.}{2006}]{WernerEtAl2006}
Werner, L, Lemos, F.O., Daudt, T. (2006) 'Previsão de demanda e níveis de estoque uma abordagem conjunta aplicada no setor siderúrgico'. {\it XIII SIMPEP}, Bauru, SP, Brasil.

\bibitem[\protect\citeauthoryear{Williams}{1984}]{Williams1984}
Williams, T.M. (1984). 'Stock control with sporadic and slow-moving demand'. {\it Journal of the Operational Research Society}, Vol. 35 (10), pp. 939–948. 

\end{thebibliography} 

\end{document}
